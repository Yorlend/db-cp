\chapter{Исследовательский раздел}

\section{Постановка исследования}

Исследуется среднее время работы различных этапов процесса добавления текста в мобильное приложение из файла формата png:

\begin{itemize}[label=---]
    \item оптическое распознавание иероглифов из изображение и преобразование их в текст;
    \item вставка полученного в результате оптического распознавания текста в базу данных;
    \item разбиение текста на отдельные слова;
    \item вставка слов, выделенных из текста, в базу данных.
\end{itemize}

Для проведения замеров использованы пять изображений из файлов формата png, примерно одинакового качества. Результатом исследования является среднее значение времени работы каждого из этапов работы разработанного программного обеспечения, по полученным результатам построена гистограмма.

\clearpage

\section{Результаты исследования}

В ходе проведения исследования был написан тестовый модуль, представленный на листинге \ref{lst:research_test.dart}.

\includelisting
    {research_test.dart}
    {Модуль для исследования характеристик ПО}

\clearpage

Гистограмма, отображающая результат исследования, показана на рисунке \ref{img:research}

\includeimage
    {research}
    {f}
    {h}
    {\linewidth}
    {Гистограмма исследования характеристик ПО}

Исходя из гистограммы, изображенной на рисунке \ref{img:research}, можно заявить, что оптическое распознавание текста занимает значительную часть времени обработки текста в приложении. Кэширование может существенно уменьшить время данного этапа работы программного обеспечения путем сохранения результатов предыдущих распознаваний и их использовании при обработке схожих или идентичных изображений. Кэширование может быть проведено как на уровне приложения, так и на уровне промежуточного слоя, хранящего результаты предыдущих распознаваний.

\clearpage

На рисунке \ref{img:research_scale} представлена зависимость времени вставки текста в базу данных от количества иероглифов в тексте.

\includeimage
    {research_scale}
    {f}
    {h}
    {\linewidth}
    {График зависимости времени вставки текста в базу данных от количества иероглифов в тексте}

\section*{Вывод из исследовательского раздела}

По полученным в ходе исследования данным было выяснено, что наибольшую часть времени работы программного обеспечения занимает оптическое распознавание иероглифов, поэтому стоит рассмотреть возможность кэширования уже распознанных иероглифов. Таким образом, при повторном обращении к уже известным изображениям, время распознавания будет сокращено. Данные исследования свидетельствуют о линейной зависимости времени вставки текста в базу данных от количества иероглифов в тексте.
