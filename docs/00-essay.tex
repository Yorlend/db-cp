\begin{essay}{}
    Цель работы: разработка мобильного приложения, предназначенного для упрощения обучения японскому языку.

    Мобильное приложение для изучения японского языка является удобным и эффективным инструментом для тех, кто хочет освоить язык. Его интерактивные функции и доступность на мобильных устройствах делают процесс изучения японского языка более гибким и увлекательным. Разработка такого приложения имеет потенциал для применения в учебных заведениях и самостоятельного обучения.

    В разрабатываемом программном обеспечении выделены три роли: пользователь, модератор и администратор. Администратор добавляет новых пользователей и изменяет данные о них. Модератор вносит тексты на оптическое распознавание, а также дает доступ к текстам другим пользователям. Пользователи могут читать тексты, получать перевод и чтение слов и добавлять их в словарь.

    Программа разработана на языке Dart с использованием фреймворка Flutter, что дает возможность разрабатывать приложение, поддерживаемое на нескольких платформах, с использованием единой кодовой базы.

    Ключевые слова: оптическое распознавание иероглифов, база данных, система управления базами данных, мобильное приложение.
\end{essay}
