\chapter{Аналитический раздел}

\section{Использование цифровых технологий в изучении языков}

Использование цифровых технологий в изучении японского языка 
является актуальным и эффективным подходом для изучения
японского языка и повышения мотивации в процессе
изучения \cite{digital-era}. Проблемы, с которыми сталкиваются
изучающие японский язык, включают необходимость запоминания большого количества новых
слов и иероглифов, а также сложности в понимании контекста
и культурных отличий.

Во время круглого стола по технологиям для усточнивого развития, организованного 
ЮНЕСКО в Париже в 2013 г.,
проведенные исследования и изученные отчеты, выделяющие значительное
преимущество применения информационных технологий в изучении языков
и в преподавании, позволили сделать вывод о главенствующей роли
цифровизации в формировании коллективных знаний \cite{japanese-comp}.

\section{Базы данных, системы управления базами данных}

База данных (БД) --- это компьютеризированная система, основное назначение
которой --- хранить информацию, предоставляя пользователям
средства ее извлечения и модификации \cite[46]{date}. Базы данных
используются для различных целей, таких как управление бизнесом,
научные исследования и медицинские записи.

Работа напрямую с базой данных может привести к нарушению целостности данных. 
Код для работы с базой данных сложен и неудобен, что может привести к ошибкам 
и проблемам с безопасностью. Избежать проблемы при работе с данными позволяет 
система управления базами данных.

Система управления базами данных (СУБД) - это программное
обеспечение, которое позволяет пользователям создавать,
управлять и обрабатывать данные в БД \cite[10--12]{pearson-subd}.
СУБД предоставляет интерфейс для работы с данными, а также обеспечивает
безопасность и целостность данных.

СУБД и БД тесно связаны, поскольку системы управления базами
данных обеспечивают доступ к данным, хранящимся в БД, и позволяют
пользователям выполнять операции с данными.

\section{Выбор системы управления базами данных}

Для разработки мобильного приложения, которое будет преобразовывать
фотографии текстов на японском языке в текстовые документы,
может быть использована как реляционная система управления
базами данных, так и нереляционная. При разработке приложения
будет использована реляционная система управления базами данных,
так как она позволяет оформить ролевую модель и хранить структурированные
данные \cite{nosql}.

Реляционные системы управления базами данных характеризуются
использованием реляционной модели управления, отличающуюся
табличной формой представления данных, а также применением
формальной математики и реляционных вычислений для обрабатываемых
данных \cite{bdbms}.

Данные, в реляционных моделях, представляют собой двумерный
массив и характеризуются следующими особенностями:

\begin{itemize}[label={$-$}]
  \item любая составляющая таблицы является одной
составляющей данных;
  \item любой столбец имеет свое уникальное имя;
  \item отсутствие одинаковых строк в таблице;
  \item все составляющие в столбцах имеют однородный тип;
  \item строки и столбцы имеют произвольный порядок \cite{relational-dbms}.
\end{itemize}

Основные современные СУБД основаны на реляционной модели данных,
в таблице \ref{tab:rel-dbms} представлен сравнительный анализ
некоторых из них.

\clearpage

\begin{table}[ht]
  \caption{Сравнительный анализ реляционных СУБД}
  \label{tab:rel-dbms}
  \begin{center}
    \begin{tabular}{|c|c|c|c|c|}
      \hline
      СУБД & Лицензия & \makecell{Масштаби- \\руемость} & \makecell{Скорость\\выполнения\\запросов} & \makecell{Управление\\транзакци-\\ями} \\
      \hline
      MSSQL \cite{mssql} & \makecell{проприе-\\тарная} & \makecell{вертикальная, \\ комплексная\\горизонтальная} & средняя & \makecell{песси-\\мистический} \\
      \hline
      MariaDB \cite{mariadb} & GNU GPL & \makecell{вертикальная\\горизонтальная} & средняя & \makecell{частично-\\опти-\\мистический} \\
      \hline
      PostgreSQL \cite{postgresql} & \makecell{открытый \\ исходный \\ код} & \makecell{вертикальная\\горизонтальная} & высокая & \makecell{частично-\\опти-\\мистический}   \\
      \hline
      Oracle \cite{oracle} & \makecell{проприе-\\тарная} & \makecell{вертикальная\\горизонтальная} & высокая & \makecell{оптимис-\\тическое} \\
      \hline
      IBM DB2 \cite{db2} & \makecell{проприе-\\тарная} & \makecell{вертикальная\\горизонтальная} & средняя & \makecell{песси-\\мистический}  \\
      \hline
      SQLite \cite{sqlite} & MIT & вертикальная & низкая & \makecell{частично-\\песси-\\мистический} \\
      \hline
    \end{tabular}
  \end{center}
\end{table}

PostgreSQL обладает высокой производительностью и безопасностью,
а также является бесплатным программным продуктом с открытым
исходным кодом \cite{relational-dbms}, что делает его доступным для использования при
разработке мобильного приложения. Кроме того, PostgreSQL имеет
хорошую поддержку хранимых процедур и триггеров, имеет хорошую
поддержку для языков SQL и Unicode, что важно для 
работы с японским языком. Например, PostgreSQL поддерживает 
полнотекстовый поиск на японском языке и имеет встроенную 
поддержку для японских иероглифов \cite{postgresql}, что делает его
хорошим выбором для разрабатываемого приложения.

\section{Обзор существующего программного обеспечения для упрощения изучения японского языка}

При изучении иностранных языков часто используются
мобильные приложения, изучение языков с поддержкой мобильных устройств
позволяет обучающемуся получить доступ к знаниям о грамматике и
лексике иностранного языка, не накладывая ограничений на место
и время изучения \cite{mobapps-ll}. Особенно информационные технологии
важны при изучении японского языка, который отличается двумя
азбуками и иероглификой, ведь система письменности является важным
аспектом при обучении любому иностранному языку.

В таблице \ref{tab:jap-apps} представлен сравнительный анализ
четырех приложений для изучения японского языка.

\begin{table}[ht]
  \caption{Обзор приложений для упрощения изучения японского языка}
  \label{tab:jap-apps}
  \begin{center}
    \begin{tabular}{|c|c|c|c|c|}
      \hline
      ПО & Duolingo \cite{duolingo} & Memrise \cite{memrise} & Lingodeer \cite{lingodeer} & WaniKani \cite{wanikani} \\
      \hline
      Цена & \makecell{Бесплатно с \\ платными \\ функциями} & \makecell{Бесплатно с\\ платными \\ функциями} & \makecell{Бесплатно с \\ платными \\ функциями} & платно \\
      \hline
      \makecell{Типы \\ упражнений} & \makecell{перевод, \\ аудирование, \\ грамматика} & \makecell{перевод, \\ аудирование, \\ грамматика} & \makecell{перевод, \\ аудирование, \\ грамматика} & \makecell{кана, \\ кандзи, \\ слова, \\ грамматика} \\
      \hline
      Платформа & \makecell{веб,\\мобильное\\приложение} & \makecell{веб,\\мобильное\\приложение} & \makecell{веб,\\мобильное\\приложение} & \makecell{веб,\\мобильное\\приложение} \\
      \hline
    \end{tabular}
  \end{center}
\end{table}

Исходя из приведенных в таблице данных, можно сделать вывод о том,
что приложения не уделяют достаточно внимания чтению текстов
на японском языке. В то же время именно чтение текстов на японском
языке представляет наибольшую сложность в процессе обучения \cite{diff-kanji}.
Для упрощения изучения японского языка в разрабатываемом мобильном
приложении будет сделан акцент на снижение сложности чтения и анализа
иероглифических текстов путем синтеза текстовых документов из
изображений отрывков из книг, журналов и других источников текстов
на японском языке.

\clearpage

\section{Проектирование базы данных}

\subsection{Диаграмма базы данных в нотации Чена}

Для проектируемой базы данных создана ER-диаграмма в нотации Чена, представленная на рисунке \ref{img:er}.

\includeimage
  {er}
  {f}
  {h}
  {\linewidth}
  {Диаграмма сущность-связь в нотации Чена}

Были выделенны 5 сущностей: \textit{иероглиф}, \textit{слово}, \textit{текст}, \textit{пользователь} и \textit{словарь}. Отношения <<многие-ко-многим>> выделены между сущностями иероглиф и слово, текст и пользователь.

\subsection{Пользовательские роли проектируемого приложения}

В разрабатываемом приложении выделяются три роли: администратор,
модератор и пользователь.

\begin{enumerate}
  \item Администратор --- имеет полный
    доступ к приложению и базе данных. Он может создавать и 
    удалять учетные записи пользователей и модераторов, а также имеет доступ к полной информации о пользователях и их активности в приложении.
  \item Модератор --- имеет ограниченный 
    доступ к приложению и базе данных. Он может создавать тексты и
    выдавать доступ к ним для пользователей. Модератор не имеет доступа к полной информации о пользователях и их активности в приложении.
  \item Пользователь --- имеет 
    доступ к основным функциям приложения, таким как изучение 
    японского языка, создание своего профиля. Пользователь не имеет полного доступа к базе данных.
\end{enumerate}

Описание пользовательских сценариев представлено на рисунке \ref{img:cw-usecase}.

\includeimage
  {cw-usecase}
  {f}
  {h}
  {0.75\linewidth}
  {Диаграмма вариантов использования приложения}

\section*{Вывод из аналитического раздела}

В данном разделе была рассмотрена предметная область и подходы
к изучению японского языка. Также были представлены основные сведения
о базах данных и системах управления базами данных для хранения
японских иероглифов и проведен анализ существующих приложений
для упрощения изучения японского языка. В ходе работы будет использоваться СУБД PostgreSQL, так как она имеет открытый исходный код, высокую производительность и безопасность, а также поддерживает хранение в базе данных иероглифов, процедур и триггеров.

При разработке приложения будет учитываться отсутствие у крупных приложений для изучения японского языка механизмов для чтения печатных текстов. Приложение будет доступно как в веб версии, так и на мобильных платформах и персональных компьютерах.
