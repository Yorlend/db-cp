\chapter{Аналитический раздел}

В данном разделе представлен обзор существующего ПО
и подходов к упрощению изучения японского языка с
помощью систем оптического распознавания текстов. Также
введены основные сведения о построении баз данных
для текстов на японском языке и словарей иероглифики.

\section{Использование цифровых технологий в изучении языков}

Использование цифровых технологий в изучении японского языка 
является актуальным и эффективным подходом для изучения
японского языка и повышения мотивации в процессе
изучения \cite{digital-era}. Проблемы, с которыми сталкиваются
изучающие японский язык, включают необходимость запоминания большого количества новых
слов и иероглифов, а также сложности в понимании контекста
и культурных отличий.

Во время круглого стола по технологиям для усточнивого развития, организованного ЮНЕСКО в Париже в 2013 г.,
проведенные исследования и изученные отчеты, выделяющие значительное
преимущество применения информационных технологий в изучении языков
и в преподавании, позволили сделать вывод о главенствующей роли
цифровизации в формировании коллективных знаний \cite{japanese-comp}.

\section{Базы данных, системы управления базами данных}

База данных (БД) --- это компьютеризированная система, основное назначение
которой --- хранить информацию, предоставляя пользователям
средства ее извлечения и модификации \cite[46]{date}. Базы данных
используются для различных целей, таких как управление бизнесом,
научные исследования и медицинские записи.

Система управления базами данных (СУБД) - это программное
обеспечение, которое позволяет пользователям создавать,
управлять и обрабатывать данные в БД \cite[10--12]{pearson-subd}.
СУБД предоставляет интерфейс для работы с данными, а также обеспечивает
безопасность и целостность данных.

