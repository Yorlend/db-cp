\chapter{Технологический раздел}

\section{Выбор средств реализации}

В качестве языка разработки был выбран Dart в сочетании с фреймворком Flutter для обеспечения кроссплатформенности приложения. Использования таких средств разработки имеет ряд преимуществ:

\begin{itemize}[label=---]
    \item одновременная разработка для различных платформ. Flutter является кроссплатформенным фреймворком, что позволяет быстро разрабатывать приложения для множества платформ с использованием единой кодовой базы;
    \item быстрая разработка интерфейса пользователя. Flutter дает возможность декларативного описания элементов интерфейса, а также располагает набором готовых графических компонентов, которые легко настраивать и комбинировать для создания интерфейса пользователя;
    \item Dart является компилируемым языком, что позволяет достичь высокого уровня производительности;
    \item Dart позволяет интегрировать сторонние библиотеки и сервисы, которые могут быть полезными при разработке.
\end{itemize}

Выбор Dart и Flutter для разработки мобильного приложения для изучения японского языка обусловлен кроссплатформенностью, быстрой разработкой интерфейса, производительностью и расширяемостью. Эти факторы помогут ускорить разработку, обеспечить хороший пользовательский опыт и упростить поддержку приложения на разных платформах.

\clearpage

\section{База данных}

Для доступа к базе данных со стороны приложения была использована технология объектно-реляционного отображения, предоставляемого библиотекой Dart-orm и PrismaORM. На листинге \ref{lst:schema.prisma} представлено описание схемы базы данных.

\includelisting
    {schema.prisma}
    {Исходный код описания схемы базы данных}

\subsection{Связь с приложением}

Со стороны приложения был использован шаблон проектирования <<репозиторий>>, предоставляющий интерфейс доступа к базе данных. Таким образом приложение не зависит от выбора конкретного способа взаимодействия с базой данных, описывая нужные для работы приложения методы, конкретная реализация которых может быть заменена. На листингах \ref{lst:user.dart}--\ref{lst:word.dart} представлен интерфейс доступа некоторым из таблиц базы данных.

\includelisting
    {user.dart}
    {Интерфейс доступа к таблице пользователей}

\includelisting
    {word.dart}
    {Интерфейс доступа к таблице слов}

\includelisting
    {text.dart}
    {Интерфейс доступа к таблице текстов}

\includelisting
    {word.dart}
    {Интерфейс доступа к таблице слов}

\clearpage

\subsection{Триггер}

Для удаления всех слов, ассоциированных только с удаляемым текстом, из словарей пользователей, была написана хранимая функция и триггер, представленные на листинге \ref{lst:trigger.sql}.

\includelisting
    {trigger.sql}
    {Описание хранимой функции и триггера}

\clearpage

\section{Аутентификация}

В рамках данной работы был написан сервис, взаимодействующий с таблицей пользователя в базе данных. Сервис, представленный на листинге~\ref{lst:auth.dart} поддерживает ролевую модель со стороны приложения.

\includelisting
    {auth.dart}
    {Сервис аутентификации}

\clearpage

\section{Тестирование}

Для тестирования функционала приложения были написаны интеграционные тесты, проверяющие взаимодействие реализаций репозиториев API, предоставляющим функционал распознавания иероглифов, разбиение на отдельные слова и поиск значений в словаре Jisho. На листингах \ref{lst:goo_test.dart}--\ref{lst:ocr_space_test.dart} представлены описанные выше интеграционные тесты.

\includelisting
    {goo_test.dart}
    {Интеграционные тесты (часть 1)}

\clearpage

\includelisting
    {jisho_test.dart}
    {Интеграционные тесты (часть 2)}

\includelisting
    {ocr_space_test.dart}
    {Интеграционные тесты (часть 3)}

Все интеграционные тесты были пройдены успешно.

\section*{Вывод из технологического раздела}

В данном разделе был выбран язык Dart в сочетании с фреймворком Flutter для реализации логики приложения и интерфейса пользователя соответственно. Также был описан интерфейс доступа к базе данных и произведено интеграционное тестирование, отражающее правильное взаимодействие разрабатываемого программного обеспечения с внешними сервисами для оптического распознавания иероглифов, разбиение на отдельные слова и поиск значений в словаре японских слов.
