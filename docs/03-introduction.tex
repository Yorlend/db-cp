\chapter*{ВВЕДЕНИЕ}
\addcontentsline{toc}{chapter}{ВВЕДЕНИЕ}

В последние годы распространена идея использования цифровых
технологий для помощи студентам в преодолении трудностей,
связанных с изучением японского языка, таких как сложная система
письма, грамматика и синтаксис, а также необходимость запоминания
большого количества новых слов и символов \cite{muhtarova}.
Также рассматривается применение сетевого обучения, как парадигмы в
дистанционном обучении для улучшения взаимодействия между
учащимися путем создания виртуальных общностей учащихся \cite{edu-network}.

Одной из наиболее важных проблем, с которым сталкивается человек
при изучении японского языка --- сложная система письма которая
включает в себя два алфавита и иероглифику. Оптическое распознавание
символов может облегчить изучение японского языка, преобразуя
изображения, содержащие японский язык, в текстовые документы,
для более легкой обработки, перевода и изучения иероглифов \cite{ocr-usage}.

Цель работы --- разработать мобильное приложение для упрощения
изучения японского языка.

Для достижения поставленной цели, необходимо выполнить следующие
задачи:

\begin{itemize}[label={$-$}]
  \item провести анализ предметной области;
  \item описать взаимодействие компонентов системы и спроектировать базу данных;
  \item описать интерфейс доступа к базе данных и произвести тестирование функционала;
  \item исследовать характеристики разработанного программного обеспечения.
\end{itemize}

