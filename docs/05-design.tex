\chapter{Конструкторский раздел}

В данном разделе представлена схема базы данных и диаграмма прецедентов. Также описано взаимодействие компонентов системы.

\section{Концептуальная модель системы}

Фундаментальной функцией разрабатываемого приложения является преобразование изображения, содержащего текст, написанный на японском языке, в текстовый документ. На рисунках \ref{img:01_A0}--\ref{img:02_A0} представлена модель системы в нотации IDEF0.

\includeimage
    {01_A0}
    {f}
    {h}
    {\linewidth}
    {Модель преобразования изображения в текст (верхний уровень)}

\clearpage

\includeimage
    {02_A0}
    {f}
    {h}
    {\linewidth}
    {Модель преобразования изображения в текст (первый уровень)}

\section{Диаграмма прецедентов}

На рисунках \ref{img:cw-user}--\ref{img:cw-admin} представлены диаграммы прецедентов.

\includeimage
    {cw-user}
    {f}
    {h}
    {0.95\linewidth}
    {Диаграмма прецедентов (пользователь)}

\includeimage
    {cw-mod}
    {f}
    {h}
    {\linewidth}
    {Диаграмма прецедентов (модератор)}

\includeimage
    {cw-admin}
    {f}
    {h}
    {\linewidth}
    {Диаграмма прецедентов (администратор)}

\clearpage

\section{Схема базы данных}

\includeimage
    {schema}
    {f}
    {h}
    {\linewidth}
    {Схема базы данных}
