\chapter{Конструкторский раздел}

\section{Концептуальная модель системы}

Фундаментальной функцией разрабатываемого приложения является преобразование изображения, содержащего текст, написанный на японском языке, в текстовый документ. Процесс преобразования изображения в текст состоит из нескольких этапов:
\begin{itemize}[label=---]
    \item проверка формата изображения;
    \item загрузка изображения для обработки;
    \item форматирование результата.
\end{itemize}
На рисунках \ref{img:01_A0}--\ref{img:02_A0} представлена модель системы в нотации IDEF0.

\includeimage
    {01_A0}
    {f}
    {h}
    {\linewidth}
    {Модель преобразования изображения в текст (верхний уровень)}

\clearpage

\includeimage
    {02_A0}
    {f}
    {h}
    {\linewidth}
    {Модель преобразования изображения в текст (первый уровень)}

\section{Диаграмма прецедентов}

На рисунках \ref{img:cw-user} представлена диаграмма прецедентов для актора <<пользователь>>.

\includeimage
    {cw-user}
    {f}
    {h}
    {0.95\linewidth}
    {Диаграмма прецедентов (пользователь)}

\clearpage

На рисунке \ref{img:cw-mod} представлена диаграмма прецедентов для актора <<модератор>>.

\includeimage
    {cw-mod}
    {f}
    {h}
    {0.95\linewidth}
    {Диаграмма прецедентов (модератор)}

На рисунке \ref{img:cw-admin} представлена диаграмма прецедентов для актора <<администратор>>.

\includeimage
    {cw-admin}
    {f}
    {h}
    {0.95\linewidth}
    {Диаграмма прецедентов (администратор)}

\clearpage

\section{Схема базы данных}

База данных содержит 7 таблиц, 2 из которые развязочные. Описание основных таблиц представлено ниже.

\begin{enumerate}
    \item Таблица <<KanjisDAO>>:
        \begin{itemize}[label=---]
            \item содержит информацию о японских иероглифах;
            \item служит основой для изучения иероглифов и обеспечивает доступ к информации о них.
        \end{itemize}
    \item Таблица <<WordsDAO>>:
        \begin{itemize}[label=---]
            \item содержит информацию о словах, состоящих из японских азбук и иероглифов;
            \item данная таблица предоставляет доступ к различным словам, их переводам и чтениям.
        \end{itemize}
    \item Таблица <<DictionariesDAO>>:
        \begin{itemize}[label=---]
            \item содержит информацию о словах, добавленных в личный словарь конкретным пользователем;
            \item таблица предоставляет доступ к сохраненным словам, для их последующего повторения.
        \end{itemize}
    \item Таблица <<TextsDAO>>:
        \begin{itemize}[label=---]
            \item содержит информацию о текстах;
            \item таблица позволяет получать информацию о тексте: его название и содержание.
        \end{itemize}
    \item Таблица <<UsersDAO>>:
        \begin{itemize}[label=---]
            \item содержит информацию о пользователях;
            \item таблица позволяет получать информацию о пользователе: его роль, логин, пароль.
        \end{itemize}
\end{enumerate}

\clearpage

На рисунке \ref{img:schema} представлена спецификация таблиц базы данных.

\includeimage
    {schema}
    {f}
    {h}
    {\linewidth}
    {Спецификация таблиц базы данных}

\section{Ограничения целостности}

На описанные таблицы будут наложены следующие ограничения целостности:

\begin{enumerate}
    \item В таблице <<KanjisDAO>> накладывается ограничение на уникальность поля \texttt{glyph}. 
    \item В таблице <<DictionariesDAO>> накладываются ограничения на уникальность полей \texttt{id}, \texttt{owner\_id}. Также поле \texttt{owner\_id} является внешним ключом для таблицы <<UsersDAO>>.
    \item В таблице <<UsersDAO>> накладываются ограничения на уникальность полей \texttt{id}, \texttt{account\_name}.
    \item В таблице <<WordsDAO>> накладывается ограничение на уникальность поля \texttt{word}.
    \item В таблице <<TextsDAO>> накладывается ограничение на уникальность поля \texttt{id}.
\end{enumerate}

\section{Требования к ПО}

К разрабатываемому ПО предъявляются следующие требования:

\begin{itemize}[label=---]
    \item приложение должно быть кроссплатформенным, поддерживать операционные системы семейства GNU/Linux, Android, а также иметь веб версию;
    \item для использования приложения необходимо пройти аутентификацию;
    \item пользователь должен иметь возможность получать информацию о словах в тексте нажатием на интересующее слово;
    \item при удалении текста модератором, все слова в словарях пользователей, относящиеся только к удаляемому тексту, также должны быть удалены.
\end{itemize}

Для удаления всех слов, связанных исключительно с удаляемым текстом, из словарей пользователей, будет использована хранимая функция и триггер в базе данных. При удалении текста модератором, триггер автоматически запускает хранимую функцию, которая удаляет все связанные слова из словарей пользователей. Это позволяет поддерживать актуальность словарей и исключать слова, которые больше не ассоциируются с каким-либо текстом.

\section*{Вывод из конструкторского раздела}

В данном разделе была определена структура базы данных с таблицами для хранения информации о пользователях, иероглифах, словах, текстах и словарях, а также связи между ними. Были определены роли пользователя, модератора, администратора и их разрешения. Для оптического преобразования изображения будет использован внешний сервис.
